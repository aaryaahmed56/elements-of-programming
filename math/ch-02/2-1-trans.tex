\begin{Lemma}
    $\mathtt{euclidean\_norm(x, y, z) = euclidean\_norm(euclidean\_norm(x, y), z)}$
\end{Lemma}

\begin{solution}
    $\mathtt{euclidean\_norm(x, y, z)}$ is a ternary operation with the signature
    $$\mathtt{euclidean\_norm :: T \times T \times T \rightarrow T}$$
    where $\mathtt{T}$ is an \textit{arithmetic type} (viz. a type on which arithmetic operations such as
    $\mathtt{+, *, \sqrt{.}}$ can be performed, e.g. unary and binary functions whose domains and codomains
    are integral or floating-point types, though not always operations: $\mathtt{\sqrt{.}}$ could possibly
    map an element from a domain of integral type to an element in a codomain of floating-point type). After
    currying, the signature becomes
    $$\mathtt{euclidean\_norm :: T \rightarrow T \rightarrow T \rightarrow T}$$
    which clearly associates to
    $$\mathtt{euclidean\_norm :: T \rightarrow \left(T \rightarrow T \rightarrow T\right)}$$

    Now, one can un-curry the parenthesized to obtain
    $$\mathtt{euclidean\_norm :: T \rightarrow \left(T \times T \rightarrow T\right)}$$
    Where, now, the parenthesized is simply the binary operation $\mathtt{euclidean\_norm(x, y)}$, which may
    be taken as a partial application of the function $\mathtt{euclidean\_norm(x,y,z)}$ to the arguments
    $\mathtt{x, y}$, upon which one can then apply the third argument $\mathtt{z}$ to obtain the Euclidean norm
    of all three. So it is possible to \enquote{build} a ternary operation from a binary operation.
\end{solution}
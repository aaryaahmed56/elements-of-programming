\begin{Lemma}
    A well-formed object is partially formed.
\end{Lemma}

\begin{solution}
    Suppose $\mathtt{a}$ is an object that is well-formed. Let $\mathtt{S_{a}}$ 
    be the state of $\mathtt{a}$, which by definition is a value $\mathtt{v : T}$ of 
    some value type $\mathtt{T}$. By well-formedness of $\mathtt{a}$, $\mathtt{S}$ is 
    also well-formed as a value, i.e. WLOG, we may reduce to the case where 
    $\mathtt{T}$ is $\mathtt{double}$ as an object type for $\mathtt{a}$. Let $\mathtt{b}$ 
    be another object of type $\mathtt{double}$. Certainly $\mathtt{a}$ may be assigned to 
    $\mathtt{b}$ without modifying the state $\mathtt{S_{b}}$ of $\mathtt{b}$, and 
    $\mathtt{a}$ may be destructed as well. Therefore $\mathtt{a}$ is partially formed.
\end{solution}
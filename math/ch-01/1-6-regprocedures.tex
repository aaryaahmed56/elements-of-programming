\begin{Exercise}
    Extend the notion of regularity to input/output objects of a procedure, 
    that is, to objects that are modified as well as read.
\end{Exercise}

\begin{solution}
    A procedure is \textit{regular} if and only if the input objects 
    $\mathtt{a_{0}}, \dots ,\mathtt{a_{r}}$, when replaced by equal objects 
    $\mathtt{b_{0}}, \dots, \mathtt{b_{r}}$, i.e. for each $i \in 
    \{0, \dots, r \}$, the states $\mathtt{S_{a_{i}}} = \mathtt{S_{b_{i}}}$ are 
    equivalent for the objects $\mathtt{a_{i}}, \mathtt{b_{i}}$ of type 
    $\mathtt{T_i}$, yield equivalent output objects $\mathtt{c_{i}} = \mathtt{d_{i}}$ 
    (where equality of objects means equivalence of the corresponding 
    states). There is a natural equivalence for input/output objects 
    in the following way: an input/output object $\mathtt{s}$ is 
    equivalent to $\mathtt{t}$, i.e. $\mathtt{s} \equiv \mathtt{t}$, if 
    there is a regular procedure $\mathtt{F}$ for which $\mathtt{t}$ is 
    the output object of the input $\mathtt{s}$ under $\mathtt{F}$, or symmetrically 
    if $\mathtt{s}$ is the output object of the input $\mathtt{t}$ under $\mathtt{F}$.
\end{solution}
\begin{Lemma}
    If a value type is uniquely represented, equality implies representational equality.
\end{Lemma}

\begin{solution}
    Suppose a value type $\mathtt{T}$ is uniquely represented. Denote by $\mathtt{v, v' : T}$ 
    as equal values of $\mathtt{T}$. By unique representation, 
    $\mathtt{v}, \mathtt{v'}$ each correspond uniquely to the abstract entities $\mathtt{E, E'}$, 
    and by equality of values, these entities must also be equal. Hence the data $\mathtt{D, D'}$ 
    for $\mathtt{v, v'}$ are identical, and so $\mathtt{v, v'}$ are representationally equal.
\end{solution}

\begin{Lemma}
    If a value type is not ambiguous, representational equality implies equality.
\end{Lemma}

\begin{solution}
    Suppose a value type $\mathtt{T}$ is not ambiguous. Denote by $\mathtt{v, v' : T}$ as 
    representationally equal values of $\mathtt{T}$. As $\mathtt{T}$ is not 
    ambiguous, $\mathtt{v, v'}$ must each have at most one interpretation, and 
    by representational equality, the data $\mathtt{D, D'}$ for the values are 
    identical. Hence the values $\mathtt{v, v'}$ must represent the same abstract 
    entity $\mathtt{E}$, and so they are equal.
\end{solution}
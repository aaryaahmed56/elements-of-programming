\begin{Lemma}
    If a value type is uniquely represented, equality implies representational equality.
\end{Lemma}

\begin{solution}
    Suppose a value type $\mathtt{T}$ is uniquely represented. Denote by $\mathtt{v},
    \mathtt{v'} : \mathtt{T}$ as equal values of $\mathtt{T}$. By unique representation, 
    $\mathtt{v}, \mathtt{v'}$ each correspond uniquely to the abstract entities $\mathtt{E}, 
    \mathtt{E'}$, and by equality of values, these entities must also be equal. 
    Hence the data $\mathtt{D}, \mathtt{D'}$ for $\mathtt{v}, \mathtt{v'}$ are 
    identical, and so $\mathtt{v}, \mathtt{v'}$ are representationally equal.
\end{solution}

\begin{Lemma}
    If a value type is not ambiguous, representational equality implies equality.
\end{Lemma}

\begin{solution}
    Suppose a value type $\mathtt{T}$ is not ambiguous. Denote by $\mathtt{v}, \mathtt{v'} : 
    \mathtt{T}$ as representationally equal values of $\mathtt{T}$. As $\mathtt{T}$ is not 
    ambiguous, $\mathtt{v}, \mathtt{v'}$ must each have at most one interpretation, and 
    by representational equality, the data $\mathtt{D}, \mathtt{D'}$ for the values are 
    identical. Hence the values $\mathtt{v}, \mathtt{v'}$ must represent the same abstract 
    entity $\mathtt{E}$, and so they are equal.
\end{solution}